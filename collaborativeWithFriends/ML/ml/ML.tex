\documentclass[12pt]{article}
\renewcommand{\baselinestretch}{1.4}

\usepackage[light,math]{iwona}
\usepackage[T1]{fontenc}

\begin{document}

\tableofcontents

\section{Foreword}
This is my English handout about Machine Learning and Deep Learning, and please excuse me from all readers for the sake of my some bad English Terminology in whole of this handout or "article". Why i choice English language for this handout? Because i will push this handout in universe repo such as GitHub or Google scholar for help someone that want learn ML and DL and etc.

\newpage

\section{What is ML}

\LARGE What is Machine Learning?\\

\small Machine Learning is a method of data analysis that automates analytical model building.\\
using algorithms that iteratively learn from data, machine learning allows computers to find hidden insights without begin explicitly programmed where to look, What is this mean? that's mean if you want create an program in Python that can tell you what you input in program, for example can say output is Blue or Red color, you maybe use some statement(if, else, elif), in ML we don't use from that!

\subsection{What is it used for?}
Machine Learning is used in a with variety of topics and use cases every things from fraud Detection to web search results to credit scoring. Or maybe you're traveling abroad and ease your credit cart and you get a call indication a possible fraud on your credit card. That's also ML attempting to detect fraudulent use cases. Then there is things like \textbf{recommendation} engines, so if you're shopping on somethings like Amazon or Digikala.com or even viewing some online streaming video service and it's recommending new videos or new movies or new TV shows to you, ML is used for that as well and things like e-mail spam filtering so the e-mails that actually go into your span folder that's using natural language processing
\footnote{NLP}to figure out what is the actual spam email and then things like pattern and image recognition. \\

Now there's certain use cases where the only possible approach is to use Deep Learning.    


\section{What are Neural Networks}
For the basics Neural Networks are way of modeling biological and you're on systems mathematically and these networks can then be used to solve tasks that many other types of algorithms can't. 
So for example that image classification it's really hard for other ML algorithms to perform well on things like image classification, and this is the kind of task where neural networks perform very well. 
So \textbf{deep learning} simply refers to neural networks with more than one hidden layer. \\

There are different types of machine learning "tasks" we will focus on during the next section.

As a quick review what is say In final note:\\
\textbf{Machine Learning}: Those are just a general terms for those automated analytical models.\\
\textbf{Neural Networks}: is actually a specific type of machine learning architecture or algorithm that specifically modeled after biological neurons.\\
\textbf{Deep Learning}: is really just a neural network with more than one hidden layer.\\ 
And we will discuss, what that means and what a hidden layer actually is in the Artificial Neural Networks in next section.\\



\section{Different between Supervised and Unsupervised Learning}

\section{Supervised learning process}

\section{Evaluate Performance}

\section{Overfitting}



\end{document}